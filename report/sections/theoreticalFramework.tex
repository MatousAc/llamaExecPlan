\section{Theoretical Framework}
\subsection{Concepts}
School websites arguing for the importance of class attendance can be found quite easily.\footnote{\href{https://www.fondafultonvilleschools.org/academics/attendance-and-academic-performance/}{fondafultonvilleschools.org}}\textsuperscript{, }\footnote{\href{https://egcsd.org/academics/attendancematters/}{egcsd.org}} According to academic research, regular attendance promises many benefits for students. Some schools even quote Woody Allen, arguing that ``80\% of success is just showing up.''\footnote{\href{https://marktomforde.com/academic/undergraduates/AttendingClass.html}{marktomforde.com}}

The National Center for Education Statistics explains that, starting in kindergarten and progressing through high school, commonly absent students miss out on learning opportunities \cite{nces2009}. As a result, even the best teacher's ability to enable student success is limited. Moreover, after leaving school, absentees ``exhibit a history of negative behaviors.''

Moreover, during the data collection phase of this project, one of the attendance system managers recalled a high school course in which their instructor announced that students would be graded based on where they sat. Students in the front rows would receive higher marks than those who chose to sit further away. In this way, the instructor was hastening what he assumed to be an inevitable outcome. Several studies, including our project, test this hypothesis.

\subsection{Similar Studies}
Researchers and scientists in various departments have conducted studies concerning some aspect of student attendance and course performance. In their 2015 study of first-year psychology courses, Alexander and Hicks analyzed whether class attendance was linked to increased student performance in modern classrooms with online lectures \cite{alexander2016}. Their results featured significant ($p<0.001$ and $p<0.05$) correlations between student attendance and performance on assignments.

Furthermore, several studies have been done concerning seat choice and student grades. In a 1973 issue of {\it Sociometry}, Becker et al. demonstrate that students sitting nearer to their instructor not only received higher grades than those further away, but also liked their professor more ($p<0.01$) \cite{becker1973}. However, other studies present contradictory conclusions \cite{kalinowski2007}.

Others have conducted experiments using machine learning to predict grades or analyze groups of students in classrooms. For example, Zabriskie et al. studied which pieces of information best predicted a student's grade in physics courses as the semester progressed \cite{zabriskie2019}. At first, a student's GPA was the strongest predicting factor, but eventually the first test grade surpassed this measure with homework performance in second place.

This project combines the interest in student attendance, seating arrangement, and machine learning algorithms to predict student grades.

