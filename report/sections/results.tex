\section{Results}
We began basic experiments by simply prompting the model to generate execution plans. As we can see in Text \ref{text:basicConvo}, there is much roo for improvement.

\begin{text}
  $<$s$>$ \#\#\# Create a query execution plan for an RDBMS for the following query: \#\#\# Query: select avg(c.credits) as 'creditAverage' from section se join course c on se.courseID = c.courseID; \#\#\# Execution Plan: [/INST]  Sure! Here's an execution plan for the query you provided:

  1. Clustered Index Scan(WHERE: (courseID = 1))
  * This is the most efficient way to access the course table, as it uses a clustered index to quickly locate the rows that match the condition.
  2. Inner Join(Seek)
  * This join is used to join the section table with the course table on the courseID column. Since $<$/s$>$$<$s$>$[INST] $<$s$>$ \#\#\# Create a query execution plan for an RDBMS for the following query: \#\#\# Query: select p.professorName, r.building, count(r.roomNumber) as 'numberOfRooms' from professor p join section se on p.professorID = se.professorID join room r on se.roomID = r.roomID group by r.building, p.professorName; \#\#\# Execution Plan: [/INST]  Great, let's create an execution plan for the query you provided:

  1. Clustered Index Scan(WHERE: (professorID = 1))
  * This is the most efficient way to access the professor table, as it uses a clustered index to quickly locate the rows that match the condition.
  2. Inner Join(Seek)
  * This join is used to join the section table with the professor table on the professorID column
  \caption{One of the earliest experiment's outputs}
  \label{text:basicConvo}
\end{text}

Although the model does begin to produce an execution plan, it always gets caught up in explaining each step and then reaches the max allowed tokens it can generate before finishing a full execution plan. To mitigate this issue, we first attempted to request the model to continue its output in another generation. We can see the results of one such example in Text \ref{text:continuedConvo}.

\begin{text}
  $<$s$>$ \#\#\# Create a query execution plan for an RDBMS for the following query: \#\#\# Query: select p.professorName, r.roomNumber from professor p join section se on p.professorID = se.professorID join room r on se.roomID = r.roomID where r.building = 'Hickman Science Center'; \#\#\# Execution Plan:
  Llama:   Sure! Here's the execution plan for the query you provided:

  1. Clustered Index Scan(WHERE: (room.building = 'Hickman Science Center'))
  2. Filter(WHERE: (se.roomID = r.roomID))
  3. Inner Join(On: (p.professorID = se.professorID))
  4. Inner Join(On: (r.roomID = se.roomID))
  Press enter to generate or type continue to prompt the model to continue: continue
  Please continue the execution plan.
  Llama:   Of course! Here's the rest of the execution plan:

  5. Sort(ORDER BY: (se.roomID ASC))
  6. Clustered Index Seek(WHERE: (se.roomID = r.roomID))
  7. Return the results

  In this execution plan, the query starts by scanning the `room` table using a clustered index scan to find all the rows that match the `building` filter. Then, it uses
  \caption{Asking the model to continue output}
  \label{text:continuedConvo}
\end{text}

When we prompted the model to continue, it successfully finished producing what it regarded as a complete execution plan. Moreover, because the execution plan is consistently formatted - it is in a numbered list - it can be extracted out of the plaintext output and reduced to a preliminary execution plan with various functions and tables.
