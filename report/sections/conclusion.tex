\section{Conclusion}
The correlation analysis did not support any correlation between a student's attendance and their performance in class. With correlation coefficients as low as $R^2=0.020$, the lack of any relationship between the independent and dependent variables is easier to argue than a correlation or causation between them.

Clustering provided more insight. A cluster of data was identified that represented tardiness or absence along with lower grades. With further work and more balanced data, this Machine Learning approach holds the most promise.

No experiments showed that sitting nearer to the front of the classroom positively impacted a student's grades. Instead, this attribute seemed generally unimportant. It only seemed useful for identifying particular students (assuming students often sat in the same seat throughout the semester).

Overall, this study does not conclusively prove correlation between student attendance and grades, nor does it prove that student seating is irrelevant to grades. It does present methods that have and have not worked well for analyzing student attendance with machine learning. Clustering, when tuned properly, could provide key insights in future studies.
