\section{Conclusion}

This research has explored the integration of large language models (LLMs) into the optimization of database query execution plans, a concept abbreviated as AI4DB. Our study focused on evaluating the capabilities of general-purpose LLMs, specifically LLaMA 2, in generating execution plans without extensive, database-specific training. The methodology involved setting up a controlled environment using a local database and prompting the LLM with structured queries to generate execution plans. The results, while showing the potential of LLMs to understand and generate basic components of execution plans, also highlighted significant gaps in accuracy and efficiency when compared to traditional SQL Server-generated plans.

The comparative analysis of LLM-generated execution plans against those produced by SQL Server revealed a notable variance in the quality and applicability of the generated plans. The LLM often included irrelevant or overly simplified steps, and lacked the specificity and logical sequence necessary for optimal query execution. These findings suggest that while LLMs can mimic certain aspects of query optimization, they currently do not match the performance and precision of dedicated query optimizers.

This study lays the groundwork for further exploration into the feasibility and effectiveness of employing LLMs in the realm of database technology, aiming to bridge the gap between AI capabilities and practical database management needs. The insights gained from this research contribute to the broader field of AI-driven database technologies, highlighting both the potential and the current limitations of using general-purpose LLMs for database query optimization.