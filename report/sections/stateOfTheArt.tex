\section{State of the Art}
In their ``AI Meets Database: AI4DB and BD4AI'' paper, Li, Zhou and Cao identify three primary ways artificial intelligence is being used to optimize database performance \cite{li2021}: cost estimation, join order, and complete optimization. Cost estimation plays a major role in join order selection, as performing joins and filters on smaller tables before larger ones decreases search times and memory usage. A query's execution plan can vary in many fundamental and nuanced characteristics, yielding millions of execution options. Thus, certain AI-based algorithms have also been used in this area to quickly arrive at an approximately optimal plan.

In 2019, Ji Sun and Guoliang Li implemented an advanced cost estimation model \cite{sun2019}. This model consisted of three layers to handle embedding, high-level query representation, and output. In order to capture the tree-like requirements of most queries and their execution plans, the model's structure was tree-based and allowed nodes to learn their subnodes' execution plans. Once this structure was achieved in the representation-layer, the estimation layer was able to produce more accurate costs than several baselines. Nevertheless, this approach suffers from needing to train a database-specific model for each application. Thus, it exchanges accuracy for generality.

Wang et al. explore another aspect relating to our research in their development of LANTERN, a tool for explaining SQL query execution plans \cite{wang2021}. While achieving an 86\% success rate in a related task, this study only describes query execution plans rather than generating them. It is useful for human learners, but not for database management systems (DBMS).

Another paper did this. It differs from what we're doing like this. Etc.
